\swapcontents
{
    % Changing babel package inside a single chapter
    % https://tex.stackexchange.com/questions/20987/changing-babel-package-inside-a-single-chapter
    %
    % Multiple-language document - babel - selectlanguage vs begin/end{otherlanguage}
    % https://tex.stackexchange.com/questions/36526/multiple-language-document-babel-selectlanguage-vs-begin-endotherlanguage
    \addtotextpreliminarycontent{English's Abstract}
    \begin{otherlanguage*}{english}
    \begin{resumo}[Abstract]

        This is the english abstract.

        \imprimirpalavraschave{Keywords}{\begin{inparaitem}[]\palavraschaveingles\end{inparaitem}}

    \end{resumo}
    \end{otherlanguage*}
}
{
    \addtotextpreliminarycontent{Resumo em Português}
    \begin{otherlanguage*}{brazil}
    \begin{resumo}[Resumo]

        Segundo a \citeonline[3.1-3.2]{NBR6028:2003}, o resumo deve ressaltar o
        objetivo, o método, os resultados e as conclusões do documento. A ordem e a extensão
        destes itens dependem do tipo de resumo (informativo ou indicativo) e do
        tratamento que cada item recebe no documento original. O resumo deve ser
        precedido da referência do documento, com exceção do resumo inserido no
        próprio documento. (\ldots) As palavras-chave devem figurar logo abaixo do
        resumo, antecedidas da expressão Palavras-chave:, separadas entre si por
        ponto e finalizadas também por ponto.

        Além disso, na UFSC o texto do resumo deve ser digitado, em um único bloco, sem espaço de parágrafo. O resumo deve
        ser significativo, composto de uma sequência de frases concisas, afirmativas e não de uma
        enumeração de tópicos. Não deve conter citações. Deve usar o verbo na voz passiva. Abaixo do
        resumo, deve-se informar as palavras-chave (palavras ou expressões significativas retiradas do
        texto) ou, termos retirados de thesaurus da área. 

        \imprimirpalavraschave{Palavras-chaves}{\begin{inparaitem}[]\palavraschaveportugues\end{inparaitem}}

    \end{resumo}
    \end{otherlanguage*}
}



% % resumo em francês
% \addtotextpreliminarycontent{Français Résumé}
% \begin{resumo}[Résumé]
%   \begin{otherlanguage*}{french}
%       Il s'agit d'un résumé en français.

%       \imprimirpalavraschave{Mots-clés}{latex. abntex. publication de textes.}
%   \end{otherlanguage*}
% \end{resumo}


% % resumo em espanhol
% \addtotextpreliminarycontent{Español Resumen}
% \begin{resumo}[Resumen]
%   \begin{otherlanguage*}{spanish}
%       Este es el resumen en español.

%       \imprimirpalavraschave{Palabras clave}{latex. abntex. publicación de textos.}
%   \end{otherlanguage*}
% \end{resumo}



