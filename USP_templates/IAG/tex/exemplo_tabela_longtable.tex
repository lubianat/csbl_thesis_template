\begin{center}
\setcaptionmargin{1cm}
\scriptsize
\begin{longtable}{lcccc}
\caption[Resumo da legenda da tabela (aparece na lista de figuras)]{Exemplo de tabela feita com o longtable.}\\
\hline \hline \\[-2ex]
\multicolumn{1}{c}{Coluna1} &
\multicolumn{1}{c}{Coluna2} &
\multicolumn{1}{c}{Coluna3} &
\multicolumn{1}{c}{Coluna4} &
\multicolumn{1}{c}{Coluna5} 

\\[0.5ex] \hline
\\[-1.8ex]

\endfirsthead

\multicolumn{5}{c}{\footnotesize{{\slshape{{\tablename} \thetable{}}} - Continuação}}\\[0.5ex]

\hline \hline\\[-2ex]

\multicolumn{1}{c}{Coluna1} &
\multicolumn{1}{c}{Coluna2} &
\multicolumn{1}{c}{Coluna3} &
\multicolumn{1}{c}{Coluna4} &
\multicolumn{1}{c}{Coluna5} 

\\[0.5ex] \hline
\\[-1.8ex]

\endhead

\multicolumn{3}{l}{{\footnotesize{Continua na próxima página\ldots}}}\\
\endfoot
\hline

\endlastfoot

1 & 2 & 3 & 4 & 5 \\
6 & 7 & 8 & 9 & 10\\

\label{tabela_com_longtable}
\end{longtable}
\end{center}